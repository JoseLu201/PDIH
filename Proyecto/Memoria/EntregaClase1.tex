\input{preambuloSimple.tex}
\usepackage{listings}

\everymath{\displaystyle}
%----------------------------------------------------------------------------------------
%	TÍTULO Y DATOS DEL ALUMNO
%----------------------------------------------------------------------------------------

\title{	
\normalfont \normalsize 
\textsc{\textbf{Periféricos y Dispositivos de Interfaz Humana} \\ Grado en Ingeniería Informática \\ Universidad de Granada} \\ [25pt] % Your university, school and/or department name(s)
\horrule{0.5pt} \\[0.4cm] % Thin top horizontal rule
\huge Reconocimiento Facial  \\ % The assignment title
\horrule{2pt} \\[0.5cm] % Thick bottom horizontal rule
\includegraphics[width=5cm]{logo}\\[8ex]
}




\author{José Luis Molina Aguilar y Sergio España Maldonado} % Nombre y apellidos

\date{\normalsize\today} % Incluye la fecha actual


%----------------------------------------------------------------------------------------
% DOCUMENTO
%----------------------------------------------------------------------------------------

\begin{document}


\maketitle % Muestra el Título
  \begin{large}
    \centering
  \vfill
  
  Curso 2022-2023\\
  Correo : joselu201@correo.ugr.es
  \vfill
  \end{large}
\newpage %inserta un salto de página

\tableofcontents % para generar el índice de contenidos

%\listoftables -----------------------------------------------------------


\newpage



%----------------------------------------------------------------------------------------
%	Cuestión 1
%----------------------------------------------------------------------------------------
\newpage
\section{Descripción}
Vamos a ver dos programas para detectar caras y otro para el reconocimiento de caras,
utilizaremos en ambos opencv, el primer programa usa algoritmo de cascada para 
detectar caras
el segundo utiliza el modulo de face recognition el cual utiliza deep learning.

\url{https://docs.opencv.org/3.4/db/d28/tutorial_cascade_classifier.html}
\section{ Deteccion de caras vs Reconocimiento facial}

La detección de caras y el reconocimiento facial son dos términos relacionados 
pero diferentes en el procesamiento de imágenes y la inteligencia artificial.\\

La detección de caras se refiere a la capacidad de detectar la presencia de una 
cara humana en una imagen o video y ubicarla dentro de la imagen. \\

El reconocimiento facial, por otro lado, se refiere a la capacidad de identificar 
una persona en función de sus características faciales. El reconocimiento facial se 
basa en la comparación de las características faciales de una persona con una base 
de datos previamente almacenada de características faciales de personas conocidas.\\

Esta técnica se utiliza en muchas aplicaciones, como la seguridad, el control 
de acceso y la identificación de personas en fotografías o videos.

\section{Modelos de cascada}
Los modelos de cascada son un tipo de algoritmo de detección de objetos que se 
utilizan comúnmente para la detección de rostros en imágenes y videos. 
La detección de cascada se basa en la utilización de una cascada de 
clasificadores que funcionan en serie, donde cada clasificador toma una 
decisión sobre la presencia o ausencia de la característica de interés en una
 determinada región de la imagen.

\begin{figure}[H]
  \centering
  \includegraphics[width=0.5\textwidth]{cascade1}
  \caption{Tipo de Característica}\label{fig:característica}
\end{figure}

Cada característica es un valor único obtenido al restar la suma de píxeles 
debajo del rectángulo blanco de la suma de píxeles debajo del rectángulo negro.

La ventaja de los modelos de cascada es que pueden ser muy rápidos y eficientes 
en la detección de objetos en imágenes y videos, lo que los hace útiles para 
aplicaciones en tiempo real. Sin embargo, también pueden tener limitaciones en 
la detección de objetos en situaciones variables de iluminación, posición y escala



\begin{figure}[H]
  \centering
  \includegraphics[width=0.5\textwidth]{cascade2}
  \caption{Ejemplo de características}\label{fig:ejemplo_característica}
\end{figure}

La fila superior muestra dos buenas características.\\
La primera característica seleccionada parece centrarse en la propiedad de 
que la región de los ojos suele ser más oscura que la región de la nariz y las mejillas.\\ 
La segunda característica seleccionada se basa en la propiedad de que los ojos son más 
oscuros que el puente de la nariz.\\


Dependiento de lo que queramos detectar, instanciaremos el clasificador de 
cascada con los diferentes modelos preentrenados que nos ofrece opencv.

\url{https://github.com/opencv/opencv/tree/4.x/data/haarcascades}

\section{Python face\_recognition}

Ahora utilizaremos un modulo de Python llamado face\_recognition
\url{https://pypi.org/project/face-recognition/}
necesitaremos tener un par de cosas instaladas antes, como cmake, dlib y visual studio.

face-recognition es una herramienta de software libre y de código abierto
que se utiliza para el reconocimiento facial y la detección de características en imágenes.\\

Proporciona una interfaz de programación de aplicaciones (API) fácil de usar
para realizar tareas de reconocimiento facial, incluyendo la detección de caras,
la extracción de características faciales y la comparación de caras para
la identificación de individuos.\\

Una vez que se han detectado las caras, la biblioteca face-recognition
utiliza un algoritmo de extracción de características faciales basado en 
"dlib" para obtener un vector de características únicas para cada cara detectada.\\
Estas características incluyen la forma de la cara, las distancias entre los ojos,
la nariz y la boca, y otros detalles únicos que se utilizan para identificar a una persona.
\end{document}
